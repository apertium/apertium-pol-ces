% File freebmt11.tex
% 17 September 2010
% Contact: freerbmt@dlsi.ua.es

\documentclass[11pt]{article}
\usepackage{freerbmt11}
\usepackage[utf8x]{inputenc}
\usepackage{times}
\usepackage{natbib}
\usepackage{url}
\urlstyle{tt}
\usepackage{latexsym}
\usepackage[T1]{fontenc}
\usepackage{multirow}

%\bibpunct{(}{)}{;}{A}{,}{,}
%\bibdata{mybib.bib}

\title{Shallow-transfer rule-based machine translation for Czech to Polish}

%\author{Joanna Ruth\\
%  Gdańsk University of Technology \\
%  Poland \\
%  \And
%  Jimmy O'Regan \\
%  Eolaistriu Technologies \\
%  Thurles \\
%  Ireland \\  
%  {\tt joregan@gmail.com}}

\author{XXX\\
  XXX \\
  XXX \\
}
\date{}

\begin{document}

\maketitle

\begin{abstract}
This article describes the development of a shallow-transfer machine translation
system from Czech to Polish in the Apertium platform. It gives details of the
methods and resources used in constructing the system. 
\end{abstract}

\section{Introduction}

Czech and Polish are both Western Slavic languages. Polish is spoken by
approximately 50 million, Czech by 12 million.
``In the 10th century, Czech and Polish were still basically the same language, 
which then began to diverge from each other, but even until the 14 century, 
Czechs and Poles understood each other without problems''\footnote{``V 10. 
století byly čeština a polština v podstatě stále jeden jazyk, pak se začaly 
od sebe rozcházet, ale ještě ve 14. století si Češi a Poláci bez problémů rozuměli.''
Authors' translation.}.~\citep{wiki:polstina}

Czech and Polish are typologically similar languages. They are both medium inflected 
languages with relatively free word order, sharing seven cases and three grammatical 
genders. In addition, both languages draw a distinction between animate and inimate 
masculine nouns, while Polish draws a further distinction with masculine animate 
nouns that refer to people\footnote{The same distinction is present in Czech, but
as it serves no role in concordance, we chose not to treat it specially.}.

Because these distinctions have a role in concordance, we chose to 
follow the example of the IPI PAN Corpus of Polish~\citep{prze:04ce} in treating these 
as separate genders. Although this greatly reduces the complexity of transfer between 
the languages, it introduces a number of artificial ambiguities.

We have been guided by earlier work on Czech--Polish rule-based machine translation~\citep{Debowski02}, 
and used issues raised by that work as a guideline in creating our own. In addition, we 
have also sought to follow the example of two previous Apertium-based projects, for Swedish to 
Danish~\citep{tyers2009rfr} and Norwegian Nynorsk and Bokm{\aa}l~\citep{unhammer2009rfr}.

As the project was begun with known time constraints, it was decided early on that we would 
concentrate on \emph{assimilation} (that is, providing an understanding of the source text). 
We also chose to focus on the Czech to Polish direction initially, as it is the direction 
we felt more comfortable with. 

Also, as there has not yet been a translation system based on Apertium between
medium inflected, free word order languages, to a certain degree we are testing
the limits of the Apertium platform.

\section{Design}
The Apertium machine translation platform uses a shallow-transfer model.
Originally designed for the Romance languages of Spain, later development has
extended the system to enable the development of translators between more
divergent language pairs, such as Basque--Spanish~\citep{ginestirosell09}.

Apertium is constructed in a modular, assembly-line fashion. 
A source language text is first morphologically analysed using finite-state 
transducers. It is then disambiguated for part of speech by a bigram HMM 
part-of-speech tagger, which produces a single disambiguated form.

The disambiguated text is subsequently processed by the syntactic transfer
module, which performs both lexical and structural transfer. Lexical units
consisting of lemma, part of speech, and morphological information are 
matched on the basis of fixed-length patterns, upon which operations such
as insertions, removals, reorderings, substitutions and concordancing are
performed.

Finally, generation
is performed by the same module that performs analysis. Figure~\ref{fig:apertium}
shows the main modules of a given system built upon the platform. A more complete description
of the platform may be found in \citet{armentano06p}.

Two models of structural transfer are supported by the platform: a single-stage transfer, where only
one set of transfer rules is used, and a three-stage transfer where rules are also used to 
group words into \emph{chunks}, on which later operations can be performed. The Czech--Polish
pair uses three-stage transfer as a means of simplifying the problems of concordance
and difference in case governed by preposition mentioned in \cite{Debowski02}.

\begin{figure*} {\small \setlength{\tabcolsep}{0.5mm}
\begin{center}
\begin{tabular}{cccccccc}
\\
\parbox{1.2cm}{SL text} $\rightarrow$ & \framebox{\parbox{1.7cm}{deformatter}} \\ 
& $\downarrow$ \\
& \framebox{\parbox{1.3cm}{morph. analyser}}  $\rightarrow$ & \framebox{\parbox{1.1cm}{POS tagger}} $\rightarrow$ & \framebox{\parbox{1.3cm}{structural\ transfer}} $\rightarrow$ & \framebox{\parbox{1.5cm}{morph. generator}}  $\rightarrow$ & \framebox{\parbox{1.3cm}{post\-generator}} & & \\
& \parbox{1.3cm}{~} & \parbox{1.1cm}{~}  & $\updownarrow$ & & $\downarrow$ & &   \\
& \parbox{1.3cm}{~} & \parbox{1.1cm}{~}  & \framebox{\parbox{1.3cm}{lexical\ transfer}} & & \framebox{\parbox{1.6cm}{reformatter}} \\
& & & & &  $\downarrow$ \\
& & & & &  \parbox{1.2cm}{TL text}\\\\
\end{tabular}
\end{center} }
\caption{The eight modules of the shallow-transfer machine translation system}
\label{fig:apertium}
\end{figure*}

\section{Development}

\subsection{Resources}

Czech and Polish are both well-resourced languages: 
``no other Slavic language has so many resources for
stochastic natural language processing'' as Czech~\citep{Hajic03}. 

There are several tools available for both languages, many open source, 
which we were able to use in the creation or validation of resources.
In particular, the Open Source Proofreading tool, LanguageTool~\citep{Milkowski10},
includes both morphological analysis and disambiguation for both Polish and
Czech.

Aside from external resources, we were able to take advantage of the Open Source nature
of the Apertium platform, and to draw upon other work being done within the project. In 
the ``incubator'' module (which houses works in various stages of early development) of 
Apertium's source repository we were able to use pieces being created for English-Polish 
and Czech-Slovenian for morphological analysis, and Polish-Slovakian for a set of initial 
transfer rules.

\begin{figure*}
\begin{small}
\begin{verbatim}
<pardef n="mat/ka__n">
  <e><p><l>ka</l><r>ka<s n="n"/><s n="f"/><s n="sg"/><s n="nom"/></r></p></e>
  <e><p><l>ki</l><r>ka<s n="n"/><s n="f"/><s n="sg"/><s n="gen"/></r></p></e>
  <e><p><l>ce</l><r>ka<s n="n"/><s n="f"/><s n="sg"/><s n="dat"/></r></p></e>
  <e><p><l>kę</l><r>ka<s n="n"/><s n="f"/><s n="sg"/><s n="acc"/></r></p></e>
  <e><p><l>ką</l><r>ka<s n="n"/><s n="f"/><s n="sg"/><s n="ins"/></r></p></e>
  <e><p><l>ce</l><r>ka<s n="n"/><s n="f"/><s n="sg"/><s n="loc"/></r></p></e>
  <e><p><l>ko</l><r>ka<s n="n"/><s n="f"/><s n="sg"/><s n="voc"/></r></p></e>
</pardef>
<pardef n="dro/ga__n">
  <e><p><l>ga</l><r>ga<s n="n"/><s n="f"/><s n="sg"/><s n="nom"/></r></p></e>
  <e><p><l>gi</l><r>ga<s n="n"/><s n="f"/><s n="sg"/><s n="gen"/></r></p></e>
  <e><p><l>dze</l><r>ga<s n="n"/><s n="f"/><s n="sg"/><s n="dat"/></r></p></e>
  <e><p><l>gę</l><r>ga<s n="n"/><s n="f"/><s n="sg"/><s n="acc"/></r></p></e>
  <e><p><l>gą</l><r>ga<s n="n"/><s n="f"/><s n="sg"/><s n="ins"/></r></p></e>
  <e><p><l>dze</l><r>ga<s n="n"/><s n="f"/><s n="sg"/><s n="loc"/></r></p></e>
  <e><p><l>go</l><r>ga<s n="n"/><s n="f"/><s n="sg"/><s n="voc"/></r></p></e>
</pardef>
\end{verbatim}
\end{small}
\caption{Paradigms for the singular parts of 
\emph{matka} and \emph{droga}, written in full form.
\label{fig:pardefs}}
\end{figure*}
\begin{figure*}
\begin{small}
\begin{verbatim}
<pardef n="BASE__matka">
  <e><p><l>a</l><r><s n="f"/><s n="sg"/><s n="nom"/></r></p></e>
  <e><p><l>i</l><r><s n="f"/><s n="sg"/><s n="gen"/></r></p></e>
  <e><p><l>ę</l><r><s n="f"/><s n="sg"/><s n="acc"/></r></p></e>
  <e><p><l>ą</l><r><s n="f"/><s n="sg"/><s n="ins"/></r></p></e>
  <e><p><l>o</l><r><s n="f"/><s n="sg"/><s n="voc"/></r></p></e>
</pardef>
<pardef n="mat/ka__n">
  <e><p><l>k</l><r>ka<s n="n"/></r></p><par n="BASE__matka"/></e>
  <e><p><l>ce</l><r>ka<s n="n"/><s n="f"/><s n="sg"/><s n="dat"/></r></p></e>
  <e><p><l>ce</l><r>ka<s n="n"/><s n="f"/><s n="sg"/><s n="loc"/></r></p></e>
</pardef>
<pardef n="dro/ga__n">
  <e><p><l>g</l><r>ga<s n="n"/></r></p><par n="BASE__matka"/></e>
  <e><p><l>dze</l><r>ga<s n="n"/><s n="f"/><s n="sg"/><s n="dat"/></r></p></e>
  <e><p><l>dze</l><r>ga<s n="n"/><s n="f"/><s n="sg"/><s n="loc"/></r></p></e>
</pardef>
\end{verbatim}
\end{small}
\caption{Paradigms for the singular parts of \emph{matka} and \emph{droga}, redefined in terms of 
a ``base paradigm'' containing the common parts of both.
\label{fig:basepardefs}}
\end{figure*}

\subsection{Morphological analysis and generation}

None of the Apertium ``incubator'' material were at a particularly advanced stage of development; in addition, both 
Czech and Polish analysis modules had serious problems that needed to be resolved. Compounding this, the 
number of distinct paradigms for nouns and verbs\footnote{\citet{Jagodzinski08} lists 174 verb 
paradigms and 146 noun paradigms; \citet{Bielec98} mentions more, and \citet{Futrega09}
contains yet more, mostly proper nouns.}, along with the number of forms each 
contained, lead to files that were difficult to edit using standard tools on an average 
computer, and made fixing inconsistencies impractical.

To overcome this, we abstracted the common parts of the open categories into sub-paradigms 
and replaced those entries with a reference to the new paradigm. Figures~\ref{fig:pardefs}
and~\ref{fig:basepardefs} illustrate this solution. Some of the paradigms from the Czech--Slovenian
translator contained a number of errors and omissions; we used the Czech Free 
Morphology\footnote{\url{http://ufal.mff.cuni.cz/pdt/Morphology_and_Tagging/Morphology/index.html}}
tool~\citep{Hacic2004} to validate entries, and to generate missing entries.

To extract entries for the monolingual dictionary, we used a similar process mentioned in \cite{Weiss05}
on the development of Lametyzator:
as both Czech and Polish ispell dictionaries were created according to linguistic principles, there 
was a one-to-one mapping between ispell ``flags'' with suffix, and paradigms in the 
morphological dictionaries.


\begin{table}
\begin{center}
 \caption{Some example mappings between Polish ispell entries and paradigms.}
 \begin{tabular}{ll}
    \hline\noalign{\smallskip}
    ispell      & Apertium\\
    \noalign{\smallskip}\hline\noalign{\smallskip}
    miłość$/$MN & miłoś$/$ć\_\_n\\
    matka$/$mMN & mat$/$ka\_\_n\\
    droga$/$mMN & fla$/$ga\_\_n\\
    \hline\noalign{\smallskip}
 \end{tabular}
\end{center}
\end{table}

\subsection{Corpus collection}
\label{sect:corpus}

Although rule-based machine translation is not corpus based, a corpus is a 
necessary tool in the empirical evaluation of rule hypotheses. We also
wished to collect a set of test sentences to use as a set of ``regression 
tests'', to ensure that additions of new rules or changes to existing rules
did not accidentally cause mistranslations which had previously been
properly handled\footnote{The set of test sentences is available on the
Apertium wiki \url{http://wiki.apertium.org/wiki/Polish_and_Czech/Pending_tests}}.

Several multilingual corpora are available that feature both
Czech and Polish, such as JRC Acquis~\citep{Steinberger2006}, 
OPUS~\citep{Tiedemann2009}, etc.; however, in these corpora
most, if not all, of the text has been translated from a third language
(usually English)\footnote{Even if there were direct translations in those 
corpora, there's no indication in the metadata of the original language.}, 
and we wished to avoid any potential ``translation drift''.

We built an initial set of sentences from the example sentences given on
the Polish edition of Wiktionary. Though there were few sentences (141), they
were mostly useful for our purposes.

We also found some material that had been translated from Czech to Polish
at an online grammar 
site\footnote{\url{http://www.finito.zanet.pl/czeski/ath/slawistyka/www.slawistyka.ath.bielsko.pl/czeski/index.html}}. 
We also found some public domain material from
the Polish embassy in Prague, which provided us with a more substantial 
set of direct translations. The Open Content parts of the collected
material has been contributed to the Open-Content Text Corpus~\citep{Banski10}, 
to contribute to the pool of freely available linguistic resources.

\subsection{Disambiguation}

For disambiguation we first chose to train a basic unsupervised bigram
part-of-speech tagger using the {\tt\small apertium-tagger} tool.
Although there are disambiguators available for both Polish and
Czech in LanguageTool, the disambiguators are not integrated with 
Apertium, and the Czech disambiguator is far from complete.

\subsection{Lexical transfer}

We were able to, with slight modifications, reuse a set of initial
transfer rules from a nascent Polish--Slovakian system\footnote{Czech 
and Slovakian are similar enough that word-to-word translation is sufficient~\citep{}.} under
development for Apertium, though the rules were quite basic
and many new rules were required.

We used several methods to create a bilingual dictionary. Closed 
categories (such as pronouns, prepositions, etc.) were added by
hand, while semi-automatic methods were used to add the open 
categories:

\begin{itemize}
\item Cognates -- we initially used the same process described in \cite{tyers2009rfr}, 
but initial results were not promising. To increase accuracy in the induced lexicon,
we restricted the process by limiting the search to words which, by their 
suffixes\footnote{\emph{-ogia, -afia} in Polish to \emph{-ogie, -afie} in Czech}, 
seemed likely Latin or Greek derivatives, to adjectives referring to 
languages\footnote{Ending in \emph{-ski} in Polish, with a corresponding \emph{-sku} form,
ending in \emph{-sky} in Czech with a corresponding \emph{-ský} form.}. While the
yield was low, the accuracy was much higher.

\item Wordlists -- we used a number of Czech-Polish wordlists, to which part-of-speech 
and gender information was added. We also used a number of Polish--English and Czech--English
wordlists via triangulation, with restrictions based on part-of-speech to filter the
resulting lists.

\item Wikipedia -- we used the process described in \cite{tyers2008}, although modified to
remove the following of interwiki links, which, although it provided ``close'' concepts
(such as providing \emph{diabetes} as a translation for \emph{diabetic}) lead to too
many incorrect entries.

\item poterminology -- we used the {\tt\small poterminology} tool, as described in
\cite{unhammer2009rfr}, with the localisation data from the KDE project, to which
we added part-of-speech information, as with the wordlists, above.

\item Probabilistic dictionary -- We trained a statistical machine translation system
using Moses \citep{Koehn2007} on the JRC Acquis Corpus \citep{Steinberger2006}. From this
we extracted the most probable translations, to which we added part-of-speech information
as with the Wordlists, above.

\item ``Stupid'' word alignment -- we passed our parallel text sources though the
analysers of both systems, taking sentences with one unknown word each, and selected
them as the probable translation, adding information as with the wordlists.
\end{itemize}

The bilingual dictionaries were manually checked at several stages of the development 
process, and bad entries were modified or discarded. 

\subsection{Syntactic transfer}

Czech and Polish are syntactically quite similar, though there are some divergences.

\begin{itemize}
\item Preposition changes -- Some prepositions take different cases in Czech and Polish,
such as \emph{pro} in Czech (accusative) to \emph{dla} in Polish (genitive). We handled
these differences as second level operations, to pass the case change to a whole noun
phrase at a time.
\item Past and conditional tenses -- Czech uses the present tense of \emph{být} (``to be'') to 
indicate person in the past and conditional tenses, while in Polish this has become
lexicalised, so that an enclitic form of the equivalent \emph{być} is considered part
of the conjugation.
\item Different modal verbs -- Czech uses the conditional particle \emph{by}, the past tense
of \emph{mít}, and the infinitive of the verb to express ``ought to'', while Polish uses
the defect modal verb \emph{powinien} with the infinitive. (However, the word for word
equivalent is used in colloquial Polish, so a rule to handle this was not strictly 
necessary for the purposes of assimilation).
\item Transgressive -- The transgressive is a verb form specific to Czech and
Slovak, which resembles a nominative-only adjective; the equivalent form in Polish is
adverbial, so we need only to discard the extra morphological information.
\end{itemize}

\section{Status}

Table~\ref{table:status} gives details of the current status of the system
in terms of the number of lemmas in each of the dictionaries and the number 
of transfer rules. The Polish dictionary contains quite a large amount of
duplicate entries, which explains much of the disparity between the numbers.

\begin{table}
\centering
\begin{tabular}{|l|c|}
\hline
                                           & Number entries\\
\hline
Monolingual dict. ({\tt pl})               & 13,973 lemmas \\
Bilingual dict.                            & 12,419 lemmas \\
Monolingual dict. ({\tt cs})               & 11,378 lemmas \\
\hline
Transfer rules ({\tt cs $\rightarrow$ pl}) &  \\
\hline
Level 1                                    & 49 \\
Level 2                                    & 20 \\
Level 3                                    & 3 \\
\hline
\end{tabular}
    \caption{Status of pair as of SVN revision 27271, 26th November 2010}
    \label{table:status}
\end{table}

\section{Evaluation}
\subsection{Coverage}

The vocabulary coverage of the system is calculated over an available corpus. Here coverage
is defined as \emph{na\"ive coverage}, that is for any given surface form at least one analysis
is returned. This may not be complete. Although we have concentrated on the Czech to Polish
direction, we also provide results for the Polish to Czech direction as an indicator of
future work.
We used the database dumps of the corresponding Wikipedias\footnote{Czech: \url{http://cs.wikipedia.org}; Access date: 30th March 2010; Filename: 
{\small\tt cswiki-20100330-pages-articles.xml.bz2}. Polish: \url{http://pl.wikipedia.org}; Access date: 4th January 2010; Filename: 
{\small\tt plwiki-20100104-pages-articles.xml.bz2}.}. The results are presented in table~\ref{table:coverage}.

\begin{table*}
\centering
\begin{tabular}{|l|c|c|c|}
\hline
Corpus & Running tokens & Known tokens & Coverage \\
\hline
Polish  & 39,293,427 & 27,997,757 & 71.25\%\\
Czech   & 17,165,777 & 10,925,926 & 63.65\%\\
\hline
\end{tabular}
    \caption{Naïve coverage for both translation directions}
    \label{table:coverage}
\end{table*}

\subsection{Quantitative}

The quantitative evaluation involved the post-edition of 46 human translated
sentences (486 words) from the Polish learner's guide to Czech. The sentences were selected from 
the portion which had been drawn from news sources. As a secondary source, we 
used the text of the UN Universal Declaration of
Human Rights (UDHR), although as the texts are not mutual translations, we can expect
a greater degree of divergence.

\begin{table*}
\centering
\begin{tabular}{|l|c|c|c|c|}
\hline
Corpus    & WER & PWER & Free rides & Unknowns\\
\hline
News Samples & 71 \% & 60 \%  & 5 \%  & 28 \% \\
UDHR & 88 \% & 68 \%  & 0 \%  & 22 \% \\
\hline
\end{tabular}
    \caption{Evaluation results for the assimilation task. Free rides are those words which
       are identical in both the source and target language. Thus although they do not cause
       a degradation in translation quality, it is relevant to take them into account when
       evaluating the system. Unknown words are included as an indication of naïve coverage
       over the test sets.}
    \label{table:quanteval}
\end{table*}

Both word error rate (WER) and position-independent error rate (PWER) were 
calculated by counting the number of insertions, substitutions and deletions
between the post-editted text and the original translation. The tool used for
calculating both WER and PWER was the freely available {\tt\small apertium-eval-translator}\footnote{The
package can be downloaded from Apertium SVN, for details see {\small \url{http://www.apertium.org/}}.}.

The results of this evaluation are shown in table~\ref{table:quanteval} and 
indicate that the system is still not ready for general use.

\subsection{Comparison}

\begin{table}
\centering
\begin{tabular}{|l|c|c|}
\hline
Corpus    & WER & PWER \\
\hline
News Samples & 76 \% & 62 \%  \\
UDHR & 47 \% & 32 \% \\
\hline
\end{tabular}
    \caption{Evaluation results for Google Translate. Free rides and Unknowns
    did not apply in this case, so were omitted.}
    \label{table:googlecompar}
\end{table}

The only other publicly available Czech to Polish system we are aware of is Google Translate. We did not
perform a full contrastive analysis, as in \cite{tyers2009rfr}, as we did not
feel the results would be particularly instructive. For the puposes of a simple
comparison, we translated the same test sets using Google Translate. The results
of the comparison are shown in table~\ref{table:googlecompar}.

The disparity in the results is somewhat surprising. It's possible that the UDHR
was part of the training set used by Google; it's also possible that Google Translate
simply performs better on material that has been translated from English, either
directly, through the use of pivot languages~\citep{Kumar07}, or indirectly, through
the use of corpora such as JRC Acquis~\citep{Steinberger2006} where both Polish and
Czech had been translated from a third language. 

\begin{table*}
\centering
\begin{tabular}{|r|l|}
\hline
Czech & Požadavky starší třiceti dnů se mažou.\\
Dębowski & Żądania starszy trzydziestu dzieni się smarują.\\
Apertium & Żądania starsze trzydziestu dni się smarują.\\
Corrected & Żądania starsze niż trzydzieści dni są wymazywane.\\
\hline
Czech & Počet dialogových procesů by měl pokrývat pracující uživatele.\\
Dębowski & Ilość dialogowych procesów miałby pokrywać pracujących użytkowników.\\
Apertium & Ilość dialogowych procesów powinien pokrywać pracujący użytkownika. \\
Corrected & Ilość procesów dialogowych powinna pokrywać ilość pracujących użytkowników.\\
\hline
\end{tabular}
    \caption{A comparison of Dębowski and our system.}
    \label{table:comparison}
\end{table*}

Although we did not have access to the system mentioned in \cite{Debowski02}, we
found it instructive to compare the performance of our system with theirs, using
the two example sentences they provide. The results
of the comparison are shown in table~\ref{table:comparison}.

In the first sentence, we have almost exactly the same output, with one difference
that can be attributed to an error in their Polish generation module. In the second
sentence, although we have the correct auxiliary verb, it is incorrectly inflected, 
because gender agreement is being taken from the wrong chunk. We also have doubly
incorrect output in the last noun chunk (\emph{pracujący użytkownika}) -- firstly,
agreement has not been performed; secondly, the singular has been generated instead
of the plural, as the result of tagging errors.

\section{Shortcomings of the system}

Although the high number of unknowns accounts for a not insignificant amount
of the errors encountered (words immediately following unknown words are treated
as having begun the sentence, causing the word to be capitalised),
most of the errors can be attributed to tagging errors. We do not consider this a problem
with {\tt\small apertium-tagger} itself, merely a reflection that the Markov
assumption that a word can be disambiguated by considering the preceding words 
does not hold sufficiently true for languages with relatively free word order.
\cite{Hajic_serial} use a combination of rule-based and statistical methods
to achieve high accuracy in Czech tagging; the rule-based component 

There are several part-of-speech taggers available for Polish~\citep{Piasecki:07TQ, Acedanski09}, 
though most are tied to a particular tagset, and not immediately suitable for use
with Apertium. We are encouraged by initial results in the rule-based 
disambiguator used in LanguageTool, which operates in a manner analogous to
Apertium's transfer system, using shallow parsing.

\cite{Debowski02} mentions the problem of differences with T-V distinction; that is,
Polish uses the words \emph{pan}, \emph{pani}, \emph{państwo} (sir, madam, ``people'')
to express polite sentiments, while Czech uses \emph{vy} (you, plural). We made no
attempt to address this.

\section{Conclusion}

We have presented results from the first free-software translator from Czech to 
Polish. This is also the first translator between free-word order languages
developed for the Apertium platform. Although the current results are far from
impressive, the problem with disambiguation is not an unsolveable one. As
work has already begun on two other language pairs with similar features, we
believe the solution to this problem will become more compelling.

For future work, there are further subcategories of words from which cognates
could be induced, as well as further sources of statistical data to be
processed. As we made no attempt to use morphological analysis or stemming
to maximise the available data, we believe our existing sources can yield
further vocabulary.

\section{Acknowledgements}

Development was funded as part of the Google Summer of Code\footnote{\url{http://code.google.com/soc/}} 
programme. 
We wish to acknowledge the debt we owe to the authors of \cite{tyers2009rfr}, which 
we used as a template in how we approached the project, as well as in this article
describing it.
We also wish to acknowledge the contributions of Petr Homola and Francis Tyers, for their
input and contributions to the development of the project.

\bibliographystyle{apalike}
\bibliography{plcs}


\end{document}
